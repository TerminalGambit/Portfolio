\documentclass[a4paper,10pt]{article}
\usepackage[left=1.5cm, right=1.5cm, top=1.5cm, bottom=1.5cm]{geometry}
\usepackage{titlesec, enumitem, hyperref, fontawesome5}
\usepackage{titlesec}
\titlespacing*{\section}{0pt}{0.5em}{0.5em}
\setlist[itemize]{noitemsep, topsep=0pt}

\titleformat{\section}{\large\bfseries}{}{0em}{}[\titlerule]
\renewcommand{\baselinestretch}{0.95} % Adjust line spacing to fit content

\begin{document}

\pagestyle{empty}

\begin{center}
    {\Huge \textbf{Jack Massey}}\\
    \vspace{2pt}
    \textbf{Etudiant en Informatique | IA et Data Science}\\
    \vspace{5pt}
    \faEnvelope \hspace{2pt} \href{mailto:jackmassey.monaco@outlook.com}{jackmassey.monaco@outlook.com} \quad
    \faPhone \hspace{2pt} +33 6 40 61 37 35 \quad
    \faLinkedin \hspace{2pt} \href{https://www.linkedin.com/in/masseyjack/}{Linkedin.com} \quad
    \faGithub \hspace{2pt} \href{https://github.com/TerminalGambit/}{GitHub.com}
\end{center}

\section*{Éducation}
\textbf{Université Côte d'Azur} -- Licence en Informatique \hfill \textit{En cours}\\
\textit{Cours pertinents :} Apprentissage Automatique, Algorithmes, IA, Science des Données\\
\textbf{Lycée Albert Ier} Baccalauréat français avec \textit{Mention Très Bien} \hfill \textit{2022}
\begin{itemize}
	\item Informatique (19/20), Mathématiques (18/20) et Physique
	\item Options Latin et Mathématiques Expertes
\end{itemize}

\section*{Compétences Techniques}
\begin{itemize}
    \item \textbf{Langages} : Python (Maîtrise), Java, C, OCaml
    \item \textbf{IA/ML} : Pandas, Scikit-learn, TensorFlow, PyTorch, Hugging Face (NLP)
    \item \textbf{Bases de Données/Web} : MySQL, PostgreSQL, Développement Web
    \item \textbf{Systèmes/Outils} : Unix, zsh, Git, GitHub, GitFlow, Agile
    \item \textbf{DevOps} : Docker, Kubernetes, Tests, CI/CD
\end{itemize}

\section*{Expérience et Projet}
\textbf{i3S Research Laboratory, Sophia Antipolis} \hfill \textit{Stagiaire d'été, 2024}\\
\textbf{Technologies :} Programmation par Contraintes, NLP, Beam Search, Divergence KL, Spécialisation\\
• Intégration de la Programmation par Contraintes et du NLP pour résoudre des problématiques concrètes\\
• Travail sur l’optimisation des processus de recherche en NLP\\
• Application de la Divergence KL et de la Perplexité pour évaluer les modèles de langage\\
• Reproduction des résultats de Yao et al., ICLR 2024, intégrés au CP 2024\\
• Lecture et analyse de publications scientifiques pour approfondir la compréhension\\
\textbf{Article de Recherche :} \href{https://arxiv.org/pdf/2407.13490}{arxiv.org/pdf/2407.13490}\\
\textbf{Travailleur Indépendant, Monaco} \hfill \textit{Tuteur Privé, 2022-Présent}\\
• Plans personnalisés en informatique, mathématiques, physique et échecs\\
• Aide à l’amélioration des compétences en résolution de problèmes et aux performances académiques\\
\textbf{Application Web – Gestion de Courses \& Recettes} \hfill 2025 \\
\textit{Angular, Symfony, PostgreSQL, API OpenFoodFacts, Docker, Raspberry Pi, Nginx} \\
App full-stack en équipe (Agile). Intégration API recettes, gestion de groupes utilisateurs. Déploiement sur Raspberry Pi avec Docker et reverse proxy via Nginx.\\
\textbf{Finance Utility – Analyse Modulaire de Marché} \hfill En cours \\
\textit{Python, Pandas, Yahoo Finance API, matplotlib/seaborn, modularité} \\
Projet personnel en cours de développement permettant d’analyser les tendances boursières (RSI, MACD, MA, Bollinger Bands). Utilisé pour renforcer mes compétences en traitement de données et finance appliquée. \href{https://github.com/TerminalGambit/PersonalStockTrackerAnalyser}{GitHub} \\
\textbf{Solo Chess – Reproduction du Jeu en Java} \hfill 2024 \\
\textit{Java, libGDX, architecture orientée objets, UML} \\
Reproduction fidèle du mode “Solo Chess” de Chess.com avec moteur de règles, logique de placement et interface graphique via libGDX. Utilisation du polymorphisme et d’un moteur de rendu modulaire. \href{https://github.com/TerminalGambit/PCOO-projetfinal}{GitHub}\\
\textbf{Serveur Multi-Clients – Communication Réseau} \hfill 2023 \\
\textit{Python, sockets, threading, signal handling} \\
Conception d’un système client-serveur multiclients en Python avec gestion concurrente par threads, envoi de signaux et handlers personnalisés. Projet en équipe de trois.\\
\textbf{Jeu en C – Interface Terminal} \hfill 2023 \\
\textit{C, gestion mémoire, logique de jeu, interface CLI} \\
Développement d’un jeu textuel interactif en C avec interface en ligne de commande, travail bas niveau sur la mémoire, accès direct, et structures de contrôle. Renforcement de la compréhension du modèle mémoire de bas niveau.

\section*{Certifications}
\begin{itemize}
    \item \textbf{Cambridge C1 English Advanced Certificate} (2021) - Maîtrise Professionnelle
\end{itemize}

\section*{Leadership et Service}
\textbf{Délégué de Classe} \hfill \textit{2018-2022}\\
• Représentation des préoccupations des élèves, facilitation de la communication avec l’administration\\
\textbf{Membre du Conseil d’Établissement} \hfill \textit{2021-2022}\\
• Participation à l’élaboration de politiques et au processus de prise de décision\\
\textbf{Journée Internationale de la Fille} \hfill \textit{Bénévole, 2022}\\
• Contribution à l’organisation d’événements visant à sensibiliser sur les droits des filles et les défis mondiaux\\
\textbf{Simulation des Nations Unies, Nice} \hfill \textit{Délégué Ambassadeur, 2018}\\
• Collaboration avec des délégués internationaux pour rédiger et adopter une résolution sur les inégalités économiques
\end{document}